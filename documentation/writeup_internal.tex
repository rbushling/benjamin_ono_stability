%This is an internal writeup. If a manuscript is written for this project, most
%of this won't go into it. 

\documentclass[a4paper,10pt]{article}
\input{macros}
\usepackage[utf8]{inputenc}
\usepackage{epstopdf}
\usepackage{enumitem}
\usepackage[title]{appendix}
\usetikzlibrary{arrows}
\usetikzlibrary{decorations.markings}
%\mathtoolsset{showonlyrefs}

%opening
\title{Benjamin-Ono stability}
\author{Ryan Bushling, Bernard Deconinck, and Jeremy Upsal
  \\
Department of Applied Mathematics,\\
University of Washington,\\
Seattle, WA 98195, USA}

\begin{document}

\maketitle

\section{Check the solution}

We consider the Benjamin-Ono equation
\begin{align}
  u_t + \mathcal{H}u_{xx} + (u^2)_x &=0,\label{eqn:BO}
\end{align}
where
\begin{align}
  \cH \lt[f(\xi)\rt](x) &= \frac1\pi \int \frac{f(\xi)}{\xi - x}~\d \xi.
\end{align}
Letting $(z,\tau) = (x-ct, t)$, so that $z$ is in the traveling frame,
\eqref{eqn:BO} becomes
\begin{align}
  u_\tau - cu_z + \cH u_{zz} + (u^2)_z &=0.\label{eqn:BOtravel}
\end{align}
We look for stationary solutions in this traveling frame
\begin{align}
  -c u_z + \cH u_{zz} + (u^2)_z &=0. \label{eqn:BOStationaryTraveling}
\end{align}
A 3-parameter family of periodic solutions to \eqref{eqn:BOStationaryTraveling}
is \cite{bronski2016modulational} given by
\begin{align}
  u(z; a, k, c) &=
  -\frac{\frac{k^2}{\sqrt{c^2-4a-k^2}}}{\sqrt{\frac{c^2-4a}{c^2-4a-k^2}} -
  \cos(kz)} + \frac12 \lt(\sqrt{c^2-4a}+c\rt),
  \label{eqn:BOsoln}
\end{align}
where $c<0$ and $k^2<c^2-4a$.  We now verify that this is indeed a solution. We
first note that \cite{ono1975}
\begin{align}
	\cH \lt(\frac{1}{1-B\cos(D\zeta)}\rt) &= \sgn(D) \frac{B
	\sin(D\zeta)}{\sqrt{1-B^2}(1-B\cos(D\zeta))},
\end{align}
and our solution may be rewritten
\begin{align}
   u(z;a,k,c) &= -\frac{k^2}{\sqrt{c^2-4a-k^2}} \sqrt{\frac{c^2-4a-k^2}{c^2-4a}}
   \frac{1}{ 1 - \sqrt{\frac{c^2-4a-k^2}{c^2-4a}}\cos(kz)} + \frac12
   \lt(\sqrt{c^2-4a}+c\rt)\\
   &= -\frac{k^2}{\sqrt{c^2-4a}}
   \frac{1}{ 1 - \sqrt{\frac{c^2-4a-k^2}{c^2-4a}}\cos(kz)} + \frac12
   \lt(\sqrt{c^2-4a}+c\rt)\\
   &= -\frac{k^2}{\alpha}\frac{1}{1-\frac{\beta}{\alpha}\cos(kz)} + \frac12(\alpha + c)\\
   &= -\frac{k^2}{\alpha}\frac{1}{1-B \cos (k z)} + \frac12( \alpha + c).
   \label{eqn:soln}
\end{align}
We then note that
\begin{align}
  \cH u_{zz} &= \cH\lt( -\partial_z^2 \lt(\frac{k^2}{\alpha}
  \frac{1}{1-B\cos(k\zeta)} + \frac12(\alpha +c)\rt)\rt)\\
  &= \cH\lt(- \partial_z^2 \lt(\frac{k^2}{\alpha}
  \frac{1}{1-B\cos(k\zeta)}\rt) \rt)\\
  &=-\frac{k^2}{\alpha} \partial_z^2 \cH\lt( 
  \frac{1}{1-B\cos(k\zeta)}\rt),
\end{align}
since two derivatives commute with the Hilbert transform. Therefore
\begin{align}
  \cH u_{zz} &= -\frac{k^2}{\alpha}\partial_z^2 \sgn(k) \frac{B
  \sin(k\zeta)}{\sqrt{1-B^2}(1-B \cos(k\zeta))}\\
  &= -\frac{k^2}{\alpha} \partial_z^2 \frac{ \beta \sgn(k)\sin(k\zeta)}{\alpha
  \sqrt{1-\beta^2/\alpha^2}(1-\beta \cos(k\zeta)/\alpha)}\\
  &= -\partial_z^2 \frac{k^2 \beta \sgn(k)\sin(k\zeta)}{\alpha\sqrt{\alpha^2-\beta^2}(1-\beta
  \cos(k\zeta)/\alpha)}.
\end{align}
But 
\begin{align}
  \alpha^2 -\beta^2 &= c^2 -4a - (c^2-4a-k^2) = k^2,
\end{align}
so that
\begin{align}
    \cH u_{zz} &= -\partial_z^2 \frac{k^2 \beta \sgn(k)\sin(k\zeta)}{\alpha
    k(1-\beta \cos(k\zeta)/\alpha)}\\
    &= -\partial_z^2 \frac{\beta k \sgn(k)\sin(k\zeta)}{\alpha - \beta \cos(k\zeta)}.
\end{align}
Then the Mathematica notebook \verb+BronskiAndHurChecks-2.nb+ shows that
\eqref{eqn:soln} is indeed a solution of \eqref{eqn:BO} and
\eqref{eqn:BOStationaryTraveling}.


\section{Fourier multiplier of the Hilbert Transform}
According to Wikipedia, the Fourier multiplier of $\cH$ is $i\sign(\omega)$
(note that the sign of the denominator in $\cH$ is different in their definition
than ours), so that
\begin{align}
  \cF[ \cH[u]](\omega) &= i\sign(\omega) \cF[u](\omega).
\end{align}
\textbf{But Wikipedia blows!} Simon says that with this definition of the
Hilbert, transform, 
\begin{align}
  \cF[ \cH[u]](\omega) &= -i\sign(\omega)\cF[u](\omega),
\end{align}
which agrees with the plotting sort of checks that Ryan has completed.

\section{Constant solution}
We can find the spectrum for the constant solution analytically, so let's do
that as a check for the numerics.

With $k=0$, \eqref{eqn:BOsoln} becomes
\begin{align}
  u(z; a,0,c) = \frac12 (\sqrt{c^2-4a}+c) =: A.
\end{align}
We linearize about this solution by letting $u(z,\tau)= A + v(z,\tau)$ where
$\abs{v}\ll1$. Plugging into \eqref{eqn:BOtravel} and retaining terms of order
$v$ and lower yields
\begin{align}
  0 &= v_\tau -c v_z + \cH v_{zz} + \partial_z(A+v)^2\\
  &= v_\tau -cv_z + \cH v_{zz} + 2A v_z.
\end{align}
Since the equation is autonomous first-order in $t$ we let $v(z,\tau) = e^{\lam
\tau}w(z)$ which yields
\begin{align}
  \lam w &= c w_z -\cH w_{zz} - 2 A w_z.\label{eqn:constantSolnStabilityODE}
\end{align}
Since the coefficients of the above ODE are periodic (since they are constant)
with period $L$, we use a Floquet decomposition,
\begin{align}
  w(z) &= e^{i\mu z}\hat w(z) ,
\end{align}
where $\mu \in [-\pi/L, \pi/L)$ and $\hat w(z+L) = \hat w(z)$. We then use an
  $L$-periodic Fourier series for $\hat w(z)$:
\begin{align}
  w(z) &= e^{i\mu z} \sum_{n\in \Z} \hat w_n e^{2\pi i n z/L} = \sum_{n\in
  \Z}\hat w_n e^{(2\pi n/L + \mu)iz}.
\end{align}
Plugging this into the ODE \eqref{eqn:constantSolnStabilityODE} yields
\begin{align}
  \lam_n &= ic \lt(\frac{2\pi n}{L} + \mu\rt) - \lt(-i\sign\lt(\frac{2\pi
  n}{L}+\mu)\rt)\rt)\lt(i\lt(\frac{2\pi n}{L}+\mu\rt)\rt)^2 - 2iA \lt(\frac{2\pi
  n}{L}+\mu\rt)\\
  &= i\lt[ -\lt(\frac{2\pi n}{L}+\mu\rt)\sqrt{c^2-4a} - \sgn\lt(\frac{2\pi
  n}{L}+\mu\rt)\lt(\frac{2\pi n}{L}+\mu\rt)^2\rt].
\end{align}
Note that
\begin{align}
  \mu + \frac{2\pi n}{L} \in \lt[\frac \pi L (2n-1), \frac \pi L(2n+1)\rt).
\end{align}
$\lam_n>0$ only if $2\pi n/L+\mu<0$, the expression becomes
\begin{align}
  \lam_{n<0} &= -i \lt[\lt(\frac{2\pi n}{L}+\mu\rt)\sqrt{c^2-4a} -
  \lt(\frac{2\pi n}L + \mu\rt)^2\rt],
\end{align}
which is maximized when $2\pi n/L+\mu$ is minimized, or when 
\begin{align}
  \frac{2\pi n}{L}+\mu = -\frac{\pi}{L}(1+2N),
\end{align}
where $N$ is the number of Fourier modes used for the truncation. Therefore
\begin{align}
  \lam_{max} &=  i \frac{\pi}{L}(1+2N)\lt[ \sqrt{c^2-4a} + \frac\pi
  L(1+2N)\rt].
\end{align}








\section{Fixing the period}
Letting 
\begin{align}
  u &= \epsilon \hat u, \qquad \hat z = \beta z, \qquad \hat \tau = \gamma \tau, \qquad
  c = \delta \hat c
\end{align}
\eqref{eqn:BOtravel} becomes
\begin{align}
  0 &= \gamma \hat u_{\hat \tau} - \delta \beta \hat c \hat u_{\hat z}+ \beta^2
  \cH [\hat u_{\hat z \hat z}] + \epsilon \beta (\hat u^2)_{\hat z}\\
  \Rightarrow 0 &= \hat u_{\hat \tau} - \frac{\delta \beta}\gamma \hat c \hat
  u_{\hat z} + \frac{\beta^2}{\gamma} \cH [\hat u_{\hat z \hat z}] +
  \frac{\epsilon \beta}{\gamma} (\hat u^2)_{\hat z}.
\end{align}
In order for this to match \eqref{eqn:BOtravel} in the traveling frame,
\begin{align}
  u = \beta \hat u, \qquad \hat z = \beta z,\qquad  \hat \tau = \beta^2 t,\qquad  c = \beta \hat c.
\end{align}
Since we want $kz = \hat z$, we take $\beta = k$. Therefore,
\begin{align}
  u = k \hat u,\qquad  \hat z = k z, \qquad \hat \tau = k^2 t, \qquad c = k \hat c,
\end{align}
and
\begin{align}
  \hat u_{\hat \tau} - \hat c \hat u_{\hat z} + \cH \hat u_{\hat z \hat z} +
  (\hat u^2)_{\hat z} &=0.\label{eqn:BOtravelScaleFree}
\end{align}
The solution \eqref{eqn:soln} becomes
\begin{align}
  \hat u &= -\frac{k}{\hat \alpha} \frac{1}{1 - \hat B \cos(\hat z)} + \frac12(\hat
  \alpha + k \hat c),
\end{align}
where
\begin{align}
  \hat \alpha = \sqrt{ k^2 \hat c^2 - 4a}, \qquad \hat B = \sqrt{\frac{ k^2 \hat
  c^2 -4a - k^2}{k^2 \hat c^2 -4a}}.
\end{align}
The mathematica notebook ``BronskiAndHurChecks-2.nb'' verifies this solution.


{\footnotesize
\bibliographystyle{siam}
\bibliography{mybib}}
\end{document}
